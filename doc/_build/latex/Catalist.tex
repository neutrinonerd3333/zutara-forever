% Generated by Sphinx.
\def\sphinxdocclass{report}
\documentclass[letterpaper,10pt,english]{sphinxmanual}
\usepackage[utf8]{inputenc}
\DeclareUnicodeCharacter{00A0}{\nobreakspace}
\usepackage{cmap}
\usepackage[T1]{fontenc}
\usepackage{babel}
\usepackage{times}
\usepackage[Bjarne]{fncychap}
\usepackage{longtable}
\usepackage{sphinx}
\usepackage{multirow}
\usepackage{amsfonts}

\addto\captionsenglish{\renewcommand{\figurename}{Fig. }}
\addto\captionsenglish{\renewcommand{\tablename}{Table }}
\SetupFloatingEnvironment{literal-block}{name=Listing }



\title{Catalist Documentation}
\date{January 18, 2016}
\release{0.0}
\author{Rachel Wu, Tony Zhang}
\newcommand{\sphinxlogo}{}
\renewcommand{\releasename}{Release}
\setcounter{tocdepth}{1}
\makeindex

\makeatletter
\def\PYG@reset{\let\PYG@it=\relax \let\PYG@bf=\relax%
    \let\PYG@ul=\relax \let\PYG@tc=\relax%
    \let\PYG@bc=\relax \let\PYG@ff=\relax}
\def\PYG@tok#1{\csname PYG@tok@#1\endcsname}
\def\PYG@toks#1+{\ifx\relax#1\empty\else%
    \PYG@tok{#1}\expandafter\PYG@toks\fi}
\def\PYG@do#1{\PYG@bc{\PYG@tc{\PYG@ul{%
    \PYG@it{\PYG@bf{\PYG@ff{#1}}}}}}}
\def\PYG#1#2{\PYG@reset\PYG@toks#1+\relax+\PYG@do{#2}}

\expandafter\def\csname PYG@tok@gd\endcsname{\def\PYG@tc##1{\textcolor[rgb]{0.63,0.00,0.00}{##1}}}
\expandafter\def\csname PYG@tok@gu\endcsname{\let\PYG@bf=\textbf\def\PYG@tc##1{\textcolor[rgb]{0.50,0.00,0.50}{##1}}}
\expandafter\def\csname PYG@tok@gt\endcsname{\def\PYG@tc##1{\textcolor[rgb]{0.00,0.27,0.87}{##1}}}
\expandafter\def\csname PYG@tok@gs\endcsname{\let\PYG@bf=\textbf}
\expandafter\def\csname PYG@tok@gr\endcsname{\def\PYG@tc##1{\textcolor[rgb]{1.00,0.00,0.00}{##1}}}
\expandafter\def\csname PYG@tok@cm\endcsname{\let\PYG@it=\textit\def\PYG@tc##1{\textcolor[rgb]{0.25,0.50,0.56}{##1}}}
\expandafter\def\csname PYG@tok@vg\endcsname{\def\PYG@tc##1{\textcolor[rgb]{0.73,0.38,0.84}{##1}}}
\expandafter\def\csname PYG@tok@vi\endcsname{\def\PYG@tc##1{\textcolor[rgb]{0.73,0.38,0.84}{##1}}}
\expandafter\def\csname PYG@tok@mh\endcsname{\def\PYG@tc##1{\textcolor[rgb]{0.13,0.50,0.31}{##1}}}
\expandafter\def\csname PYG@tok@cs\endcsname{\def\PYG@tc##1{\textcolor[rgb]{0.25,0.50,0.56}{##1}}\def\PYG@bc##1{\setlength{\fboxsep}{0pt}\colorbox[rgb]{1.00,0.94,0.94}{\strut ##1}}}
\expandafter\def\csname PYG@tok@ge\endcsname{\let\PYG@it=\textit}
\expandafter\def\csname PYG@tok@vc\endcsname{\def\PYG@tc##1{\textcolor[rgb]{0.73,0.38,0.84}{##1}}}
\expandafter\def\csname PYG@tok@il\endcsname{\def\PYG@tc##1{\textcolor[rgb]{0.13,0.50,0.31}{##1}}}
\expandafter\def\csname PYG@tok@go\endcsname{\def\PYG@tc##1{\textcolor[rgb]{0.20,0.20,0.20}{##1}}}
\expandafter\def\csname PYG@tok@cp\endcsname{\def\PYG@tc##1{\textcolor[rgb]{0.00,0.44,0.13}{##1}}}
\expandafter\def\csname PYG@tok@gi\endcsname{\def\PYG@tc##1{\textcolor[rgb]{0.00,0.63,0.00}{##1}}}
\expandafter\def\csname PYG@tok@gh\endcsname{\let\PYG@bf=\textbf\def\PYG@tc##1{\textcolor[rgb]{0.00,0.00,0.50}{##1}}}
\expandafter\def\csname PYG@tok@ni\endcsname{\let\PYG@bf=\textbf\def\PYG@tc##1{\textcolor[rgb]{0.84,0.33,0.22}{##1}}}
\expandafter\def\csname PYG@tok@nl\endcsname{\let\PYG@bf=\textbf\def\PYG@tc##1{\textcolor[rgb]{0.00,0.13,0.44}{##1}}}
\expandafter\def\csname PYG@tok@nn\endcsname{\let\PYG@bf=\textbf\def\PYG@tc##1{\textcolor[rgb]{0.05,0.52,0.71}{##1}}}
\expandafter\def\csname PYG@tok@no\endcsname{\def\PYG@tc##1{\textcolor[rgb]{0.38,0.68,0.84}{##1}}}
\expandafter\def\csname PYG@tok@na\endcsname{\def\PYG@tc##1{\textcolor[rgb]{0.25,0.44,0.63}{##1}}}
\expandafter\def\csname PYG@tok@nb\endcsname{\def\PYG@tc##1{\textcolor[rgb]{0.00,0.44,0.13}{##1}}}
\expandafter\def\csname PYG@tok@nc\endcsname{\let\PYG@bf=\textbf\def\PYG@tc##1{\textcolor[rgb]{0.05,0.52,0.71}{##1}}}
\expandafter\def\csname PYG@tok@nd\endcsname{\let\PYG@bf=\textbf\def\PYG@tc##1{\textcolor[rgb]{0.33,0.33,0.33}{##1}}}
\expandafter\def\csname PYG@tok@ne\endcsname{\def\PYG@tc##1{\textcolor[rgb]{0.00,0.44,0.13}{##1}}}
\expandafter\def\csname PYG@tok@nf\endcsname{\def\PYG@tc##1{\textcolor[rgb]{0.02,0.16,0.49}{##1}}}
\expandafter\def\csname PYG@tok@si\endcsname{\let\PYG@it=\textit\def\PYG@tc##1{\textcolor[rgb]{0.44,0.63,0.82}{##1}}}
\expandafter\def\csname PYG@tok@s2\endcsname{\def\PYG@tc##1{\textcolor[rgb]{0.25,0.44,0.63}{##1}}}
\expandafter\def\csname PYG@tok@nt\endcsname{\let\PYG@bf=\textbf\def\PYG@tc##1{\textcolor[rgb]{0.02,0.16,0.45}{##1}}}
\expandafter\def\csname PYG@tok@nv\endcsname{\def\PYG@tc##1{\textcolor[rgb]{0.73,0.38,0.84}{##1}}}
\expandafter\def\csname PYG@tok@s1\endcsname{\def\PYG@tc##1{\textcolor[rgb]{0.25,0.44,0.63}{##1}}}
\expandafter\def\csname PYG@tok@ch\endcsname{\let\PYG@it=\textit\def\PYG@tc##1{\textcolor[rgb]{0.25,0.50,0.56}{##1}}}
\expandafter\def\csname PYG@tok@m\endcsname{\def\PYG@tc##1{\textcolor[rgb]{0.13,0.50,0.31}{##1}}}
\expandafter\def\csname PYG@tok@gp\endcsname{\let\PYG@bf=\textbf\def\PYG@tc##1{\textcolor[rgb]{0.78,0.36,0.04}{##1}}}
\expandafter\def\csname PYG@tok@sh\endcsname{\def\PYG@tc##1{\textcolor[rgb]{0.25,0.44,0.63}{##1}}}
\expandafter\def\csname PYG@tok@ow\endcsname{\let\PYG@bf=\textbf\def\PYG@tc##1{\textcolor[rgb]{0.00,0.44,0.13}{##1}}}
\expandafter\def\csname PYG@tok@sx\endcsname{\def\PYG@tc##1{\textcolor[rgb]{0.78,0.36,0.04}{##1}}}
\expandafter\def\csname PYG@tok@bp\endcsname{\def\PYG@tc##1{\textcolor[rgb]{0.00,0.44,0.13}{##1}}}
\expandafter\def\csname PYG@tok@c1\endcsname{\let\PYG@it=\textit\def\PYG@tc##1{\textcolor[rgb]{0.25,0.50,0.56}{##1}}}
\expandafter\def\csname PYG@tok@o\endcsname{\def\PYG@tc##1{\textcolor[rgb]{0.40,0.40,0.40}{##1}}}
\expandafter\def\csname PYG@tok@kc\endcsname{\let\PYG@bf=\textbf\def\PYG@tc##1{\textcolor[rgb]{0.00,0.44,0.13}{##1}}}
\expandafter\def\csname PYG@tok@c\endcsname{\let\PYG@it=\textit\def\PYG@tc##1{\textcolor[rgb]{0.25,0.50,0.56}{##1}}}
\expandafter\def\csname PYG@tok@mf\endcsname{\def\PYG@tc##1{\textcolor[rgb]{0.13,0.50,0.31}{##1}}}
\expandafter\def\csname PYG@tok@err\endcsname{\def\PYG@bc##1{\setlength{\fboxsep}{0pt}\fcolorbox[rgb]{1.00,0.00,0.00}{1,1,1}{\strut ##1}}}
\expandafter\def\csname PYG@tok@mb\endcsname{\def\PYG@tc##1{\textcolor[rgb]{0.13,0.50,0.31}{##1}}}
\expandafter\def\csname PYG@tok@ss\endcsname{\def\PYG@tc##1{\textcolor[rgb]{0.32,0.47,0.09}{##1}}}
\expandafter\def\csname PYG@tok@sr\endcsname{\def\PYG@tc##1{\textcolor[rgb]{0.14,0.33,0.53}{##1}}}
\expandafter\def\csname PYG@tok@mo\endcsname{\def\PYG@tc##1{\textcolor[rgb]{0.13,0.50,0.31}{##1}}}
\expandafter\def\csname PYG@tok@kd\endcsname{\let\PYG@bf=\textbf\def\PYG@tc##1{\textcolor[rgb]{0.00,0.44,0.13}{##1}}}
\expandafter\def\csname PYG@tok@mi\endcsname{\def\PYG@tc##1{\textcolor[rgb]{0.13,0.50,0.31}{##1}}}
\expandafter\def\csname PYG@tok@kn\endcsname{\let\PYG@bf=\textbf\def\PYG@tc##1{\textcolor[rgb]{0.00,0.44,0.13}{##1}}}
\expandafter\def\csname PYG@tok@cpf\endcsname{\let\PYG@it=\textit\def\PYG@tc##1{\textcolor[rgb]{0.25,0.50,0.56}{##1}}}
\expandafter\def\csname PYG@tok@kr\endcsname{\let\PYG@bf=\textbf\def\PYG@tc##1{\textcolor[rgb]{0.00,0.44,0.13}{##1}}}
\expandafter\def\csname PYG@tok@s\endcsname{\def\PYG@tc##1{\textcolor[rgb]{0.25,0.44,0.63}{##1}}}
\expandafter\def\csname PYG@tok@kp\endcsname{\def\PYG@tc##1{\textcolor[rgb]{0.00,0.44,0.13}{##1}}}
\expandafter\def\csname PYG@tok@w\endcsname{\def\PYG@tc##1{\textcolor[rgb]{0.73,0.73,0.73}{##1}}}
\expandafter\def\csname PYG@tok@kt\endcsname{\def\PYG@tc##1{\textcolor[rgb]{0.56,0.13,0.00}{##1}}}
\expandafter\def\csname PYG@tok@sc\endcsname{\def\PYG@tc##1{\textcolor[rgb]{0.25,0.44,0.63}{##1}}}
\expandafter\def\csname PYG@tok@sb\endcsname{\def\PYG@tc##1{\textcolor[rgb]{0.25,0.44,0.63}{##1}}}
\expandafter\def\csname PYG@tok@k\endcsname{\let\PYG@bf=\textbf\def\PYG@tc##1{\textcolor[rgb]{0.00,0.44,0.13}{##1}}}
\expandafter\def\csname PYG@tok@se\endcsname{\let\PYG@bf=\textbf\def\PYG@tc##1{\textcolor[rgb]{0.25,0.44,0.63}{##1}}}
\expandafter\def\csname PYG@tok@sd\endcsname{\let\PYG@it=\textit\def\PYG@tc##1{\textcolor[rgb]{0.25,0.44,0.63}{##1}}}

\def\PYGZbs{\char`\\}
\def\PYGZus{\char`\_}
\def\PYGZob{\char`\{}
\def\PYGZcb{\char`\}}
\def\PYGZca{\char`\^}
\def\PYGZam{\char`\&}
\def\PYGZlt{\char`\<}
\def\PYGZgt{\char`\>}
\def\PYGZsh{\char`\#}
\def\PYGZpc{\char`\%}
\def\PYGZdl{\char`\$}
\def\PYGZhy{\char`\-}
\def\PYGZsq{\char`\'}
\def\PYGZdq{\char`\"}
\def\PYGZti{\char`\~}
% for compatibility with earlier versions
\def\PYGZat{@}
\def\PYGZlb{[}
\def\PYGZrb{]}
\makeatother

\renewcommand\PYGZsq{\textquotesingle}

\begin{document}

\maketitle
\tableofcontents
\phantomsection\label{index::doc}


Contents:
\phantomsection\label{index:module-routes}\index{routes (module)}\index{Catalist (class in routes)}

\begin{fulllineitems}
\phantomsection\label{index:routes.Catalist}\pysiglinewithargsret{\strong{class }\code{routes.}\bfcode{Catalist}}{\emph{*args}, \emph{**values}}{}
A class for our lists (Catalists :P)

\end{fulllineitems}

\index{CatalistEntry (class in routes)}

\begin{fulllineitems}
\phantomsection\label{index:routes.CatalistEntry}\pysiglinewithargsret{\strong{class }\code{routes.}\bfcode{CatalistEntry}}{\emph{*args}, \emph{**kwargs}}{}
A class for the entries in our Catalists

\end{fulllineitems}

\index{CatalistKVP (class in routes)}

\begin{fulllineitems}
\phantomsection\label{index:routes.CatalistKVP}\pysiglinewithargsret{\strong{class }\code{routes.}\bfcode{CatalistKVP}}{\emph{*args}, \emph{**kwargs}}{}
A class for individual key-value pairs in our Catalist entries

\end{fulllineitems}

\index{InvalidAPIUsage}

\begin{fulllineitems}
\phantomsection\label{index:routes.InvalidAPIUsage}\pysiglinewithargsret{\strong{exception }\code{routes.}\bfcode{InvalidAPIUsage}}{\emph{message}, \emph{status\_code=None}, \emph{payload=None}}{}
A class for exceptions to raise in invalid API usage.
Shamelessly pillaged from \href{http://flask.pocoo.org/docs/0.10/patterns/apierrors/}{Flask's documentation}

\end{fulllineitems}

\index{Role (class in routes)}

\begin{fulllineitems}
\phantomsection\label{index:routes.Role}\pysiglinewithargsret{\strong{class }\code{routes.}\bfcode{Role}}{\emph{*args}, \emph{**values}}{}
A class for user roles (e.g. user, admin, ...)

\end{fulllineitems}

\index{User (class in routes)}

\begin{fulllineitems}
\phantomsection\label{index:routes.User}\pysiglinewithargsret{\strong{class }\code{routes.}\bfcode{User}}{\emph{*args}, \emph{**values}}{}
A class for users. Can have any/none of these attributes.

\end{fulllineitems}

\index{about() (in module routes)}

\begin{fulllineitems}
\phantomsection\label{index:routes.about}\pysiglinewithargsret{\code{routes.}\bfcode{about}}{}{}
About us.

\end{fulllineitems}

\index{admin\_unames (in module routes)}

\begin{fulllineitems}
\phantomsection\label{index:routes.admin_unames}\pysigline{\code{routes.}\bfcode{admin\_unames}\strong{ = {[}'rmwu', `txz'{]}}}
Currently admins are determined by residency on this list.
Hacky, I know. .\_\_.

\end{fulllineitems}

\index{autocomplete() (in module routes)}

\begin{fulllineitems}
\phantomsection\label{index:routes.autocomplete}\pysiglinewithargsret{\code{routes.}\bfcode{autocomplete}}{}{}
completes a word fragment with a possible list type
usage: POST to this route with \{``fragment'': myfragment\},
response is the list of possible completions of \emph{myfragment}
drawn from \emph{autocomplete\_dict}

\end{fulllineitems}

\index{cmp\_permission() (in module routes)}

\begin{fulllineitems}
\phantomsection\label{index:routes.cmp_permission}\pysiglinewithargsret{\code{routes.}\bfcode{cmp\_permission}}{\emph{perm1}, \emph{perm2}}{}
Return a positive/0/negative integer when perm1 \textgreater{}/=/\textless{} perm2

\end{fulllineitems}

\index{create\_list() (in module routes)}

\begin{fulllineitems}
\phantomsection\label{index:routes.create_list}\pysiglinewithargsret{\code{routes.}\bfcode{create\_list}}{}{}
Create a new list and return the assigned listid

Returns: the assigned listid

\end{fulllineitems}

\index{entry\_delete() (in module routes)}

\begin{fulllineitems}
\phantomsection\label{index:routes.entry_delete}\pysiglinewithargsret{\code{routes.}\bfcode{entry\_delete}}{}{}
Delete an entry from a Catalist. Requires at least edit permission.

usage: POST a JSON associative array as follows:
\{
\begin{quote}

listid: \textless{}the id of the Catalist\textgreater{},
entryind: \textless{}the index of the entry to remove\textgreater{}
\end{quote}

\}

\end{fulllineitems}

\index{entry\_title\_save() (in module routes)}

\begin{fulllineitems}
\phantomsection\label{index:routes.entry_title_save}\pysiglinewithargsret{\code{routes.}\bfcode{entry\_title\_save}}{}{}
AJAXily save the title of an entry. Requires at least edit permission

usage: POST a JS associative array (basically a dict) like so:
\{
\begin{quote}

listid:  \textless{}the list id\textgreater{},
entryind: \textless{}index of entry w.r.t. list (0-indexing)\textgreater{},
newvalue: \textless{}new entry title\textgreater{}
\end{quote}

\}

\end{fulllineitems}

\index{get\_id() (in module routes)}

\begin{fulllineitems}
\phantomsection\label{index:routes.get_id}\pysiglinewithargsret{\code{routes.}\bfcode{get\_id}}{}{}
Return name of current user

\end{fulllineitems}

\index{getlist() (in module routes)}

\begin{fulllineitems}
\phantomsection\label{index:routes.getlist}\pysiglinewithargsret{\code{routes.}\bfcode{getlist}}{\emph{listid}}{}
Display a list with given listid from our database.

\end{fulllineitems}

\index{human\_readable\_time\_since() (in module routes)}

\begin{fulllineitems}
\phantomsection\label{index:routes.human_readable_time_since}\pysiglinewithargsret{\code{routes.}\bfcode{human\_readable\_time\_since}}{\emph{tiem}}{}
Give a human-readable representation of time elapsed since a given time
\begin{quote}\begin{description}
\item[{Parameters}] \leavevmode
\textbf{\texttt{tiem}} -- a \code{datetime} object representing the given time.

\end{description}\end{quote}

\end{fulllineitems}

\index{index() (in module routes)}

\begin{fulllineitems}
\phantomsection\label{index:routes.index}\pysiglinewithargsret{\code{routes.}\bfcode{index}}{}{}
Our homepage!

\end{fulllineitems}

\index{key\_save() (in module routes)}

\begin{fulllineitems}
\phantomsection\label{index:routes.key_save}\pysiglinewithargsret{\code{routes.}\bfcode{key\_save}}{}{}
Save a key. Requires at least edit permission.

POST a JS associative array (basically a dict) like so:
\{
\begin{quote}

listid:  \textless{}the list id\textgreater{},
entryind: \textless{}index of entry w.r.t. list (0-indexing)\textgreater{},
index: \textless{}index of key-val pair w.r.t. entry\textgreater{},
newvalue: \textless{}new value of key\textgreater{}
\end{quote}

\}

\end{fulllineitems}

\index{kvp\_delete() (in module routes)}

\begin{fulllineitems}
\phantomsection\label{index:routes.kvp_delete}\pysiglinewithargsret{\code{routes.}\bfcode{kvp\_delete}}{}{}
Delete a key-value pair from a Catalist entry.
Requires at least edit permission.

usage: POST a JSON associative array as follows:
\{
\begin{quote}

listid: \textless{}the id of the Catalist\textgreater{},
entryind: \textless{}the index of the entry to remove\textgreater{},
index: \textless{}the index of the kvp within the entry\textgreater{}
\end{quote}

\}

\end{fulllineitems}

\index{list\_delete() (in module routes)}

\begin{fulllineitems}
\phantomsection\label{index:routes.list_delete}\pysiglinewithargsret{\code{routes.}\bfcode{list\_delete}}{}{}
Delete a Catalist. Requires at least own permission

usage: POST a JSON associative array as follows:
\{
\begin{quote}

listid: \textless{}the id of the list to be deleted\textgreater{}
\end{quote}

\}

\end{fulllineitems}

\index{list\_save() (in module routes)}

\begin{fulllineitems}
\phantomsection\label{index:routes.list_save}\pysiglinewithargsret{\code{routes.}\bfcode{list\_save}}{}{}
Save an entire list. If listid is provided, the list is
written onto the referenced list. Otherwise, a new list is
created. In both cases the listid to which we saved the list
is returned.

usage:
\{
\begin{quote}

title: \textless{}thetitle\textgreater{},
contents: {[}
\begin{quote}
\begin{description}
\item[{{[}title, {[}}] \leavevmode
{[}attrname, attrval{]},
...
{]}

\end{description}
\end{quote}

{]}
(optionally) , listid: \textless{}the listid to save to\textgreater{}
\end{quote}

\}

Returns: the given or assigned listid

\end{fulllineitems}

\index{list\_title\_save() (in module routes)}

\begin{fulllineitems}
\phantomsection\label{index:routes.list_title_save}\pysiglinewithargsret{\code{routes.}\bfcode{list\_title\_save}}{}{}
AJAXily save the title of a Catalist \textasciicircum{}\_\textasciicircum{}
Requires at least edit permission.

usage: POST a JS assoc array like so:
\{
\begin{quote}

listid: \textless{}the list id\textgreater{},
newvalue: \textless{}our new title\textgreater{}
\end{quote}

\}

\end{fulllineitems}

\index{login() (in module routes)}

\begin{fulllineitems}
\phantomsection\label{index:routes.login}\pysiglinewithargsret{\code{routes.}\bfcode{login}}{}{}
Page for user login

\end{fulllineitems}

\index{logout() (in module routes)}

\begin{fulllineitems}
\phantomsection\label{index:routes.logout}\pysiglinewithargsret{\code{routes.}\bfcode{logout}}{}{}
Log out the current user, clearing the Remember Me cookie

\end{fulllineitems}

\index{make\_list() (in module routes)}

\begin{fulllineitems}
\phantomsection\label{index:routes.make_list}\pysiglinewithargsret{\code{routes.}\bfcode{make\_list}}{}{}
Upon making the first edit, an empty list will be
created for the insertion of more data

\end{fulllineitems}

\index{perm\_list (in module routes)}

\begin{fulllineitems}
\phantomsection\label{index:routes.perm_list}\pysigline{\code{routes.}\bfcode{perm\_list}\strong{ = {[}'none', `view', `edit', `own', `admin'{]}}}
A list of all permission levels, from lowest to highest.
The levels:
\begin{enumerate}
\item {} 
none  -- no permission

\item {} 
view  -- permission to view a list

\item {} 
edit  -- permission to edit a list

\item {} 
own   -- permission to change permission for a list

\item {} 
admin -- can do anything

\end{enumerate}

\end{fulllineitems}

\index{permissions\_get() (in module routes)}

\begin{fulllineitems}
\phantomsection\label{index:routes.permissions_get}\pysiglinewithargsret{\code{routes.}\bfcode{permissions\_get}}{}{}
Get the permission level a user has for a particular list.

usage: POST the following:
\{
\begin{quote}

listid: \textless{}the listid\textgreater{}
\end{quote}

\}

returns:
\{
\begin{quote}

permission: \textless{}the current permission\textgreater{}
\end{quote}

\}

\end{fulllineitems}

\index{permissions\_set() (in module routes)}

\begin{fulllineitems}
\phantomsection\label{index:routes.permissions_set}\pysiglinewithargsret{\code{routes.}\bfcode{permissions\_set}}{}{}~\begin{description}
\item[{\{}] \leavevmode
Set permissions for a user. Requires at least own permission

listid: \textless{}listid\textgreater{},
target: \textless{}username of user to set perms with\textgreater{},
permission: \{none \textbar{} view \textbar{} edit \textbar{} own \textbar{} admin\}

\end{description}

\}

\end{fulllineitems}

\index{preview\_list() (in module routes)}

\begin{fulllineitems}
\phantomsection\label{index:routes.preview_list}\pysiglinewithargsret{\code{routes.}\bfcode{preview\_list}}{\emph{listid}}{}
Fetch the list with given listid from our database,
display with template

\end{fulllineitems}

\index{query\_cur\_perm() (in module routes)}

\begin{fulllineitems}
\phantomsection\label{index:routes.query_cur_perm}\pysiglinewithargsret{\code{routes.}\bfcode{query\_cur\_perm}}{\emph{catalist}}{}
Finds the permission the current user has for list \emph{catalist}

\end{fulllineitems}

\index{query\_permission() (in module routes)}

\begin{fulllineitems}
\phantomsection\label{index:routes.query_permission}\pysiglinewithargsret{\code{routes.}\bfcode{query\_permission}}{\emph{user}, \emph{catalist}}{}
Gives the permission level a user has for a list.
``None'' represents an anonymous user.

\end{fulllineitems}

\index{register() (in module routes)}

\begin{fulllineitems}
\phantomsection\label{index:routes.register}\pysiglinewithargsret{\code{routes.}\bfcode{register}}{}{}
Page for user registration

\end{fulllineitems}

\index{signin() (in module routes)}

\begin{fulllineitems}
\phantomsection\label{index:routes.signin}\pysiglinewithargsret{\code{routes.}\bfcode{signin}}{}{}
Sign the user in, given valid credentials.

\end{fulllineitems}

\index{signup() (in module routes)}

\begin{fulllineitems}
\phantomsection\label{index:routes.signup}\pysiglinewithargsret{\code{routes.}\bfcode{signup}}{}{}
Sign the user up, given valid credentials and a username the doesn't
already exist in our database.

\end{fulllineitems}

\index{userlists() (in module routes)}

\begin{fulllineitems}
\phantomsection\label{index:routes.userlists}\pysiglinewithargsret{\code{routes.}\bfcode{userlists}}{\emph{*args}, \emph{**kwargs}}{}
A page displaying all lists belonging to the user.

\end{fulllineitems}

\index{value\_save() (in module routes)}

\begin{fulllineitems}
\phantomsection\label{index:routes.value_save}\pysiglinewithargsret{\code{routes.}\bfcode{value\_save}}{}{}
Save the value in a particular key-value pair. Requires
at least edit permission.

The API is virtually identical the that of key\_save()

\end{fulllineitems}

\index{vote() (in module routes)}

\begin{fulllineitems}
\phantomsection\label{index:routes.vote}\pysiglinewithargsret{\code{routes.}\bfcode{vote}}{}{}
Two options:
1. Update the database to incorporate a user's vote on an entry.
2. Find the user's current vote and the current score of the entry.
Requires at least view permission

usage: POST the following
\{
\begin{quote}

listid: \textless{}listid\textgreater{},
entryind: \textless{}entryind\textgreater{},
userid: \textless{}userid\textgreater{},
vote: \{1 (upvote) \textbar{} 0 (no vote) \textbar{}
\begin{quote}

-1 (downvote) \textbar{} 100 (get the current vote)\}
\end{quote}
\end{quote}

\}
\begin{quote}\begin{description}
\item[{Returns}] \leavevmode
a response with the following forms:

\end{description}\end{quote}
\begin{description}
\item[{if vote == 100 \{}] \leavevmode
current\_vote: \textless{}the user's current vote\textgreater{},
score: \textless{}the entry's current score\textgreater{}

\end{description}

\}
if vote != 100 \{
\begin{quote}

current\_vote: \textless{}the vote just made\textgreater{},
score: \textless{}the entry's new score\textgreater{}
\end{quote}

\}

\end{fulllineitems}



\chapter{Indices and tables}
\label{index:welcome-to-catalist-s-documentation}\label{index:indices-and-tables}\begin{itemize}
\item {} 
\DUspan{xref,std,std-ref}{genindex}

\item {} 
\DUspan{xref,std,std-ref}{modindex}

\item {} 
\DUspan{xref,std,std-ref}{search}

\end{itemize}


\renewcommand{\indexname}{Python Module Index}
\begin{theindex}
\def\bigletter#1{{\Large\sffamily#1}\nopagebreak\vspace{1mm}}
\bigletter{r}
\item {\texttt{routes}}, \pageref{index:module-routes}
\end{theindex}

\renewcommand{\indexname}{Index}
\printindex
\end{document}
